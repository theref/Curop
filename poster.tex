\documentclass[a0,landscape]{a0poster}

\usepackage{multicol}
\columnsep=100pt
\columnseprule=3pt

\usepackage[svgnames]{xcolor}

\usepackage[scaled]{helvet}
\renewcommand\familydefault{\sfdefault}
\usepackage[T1]{fontenc}

\usepackage{float}
\usepackage{tikz}
\usetikzlibrary{patterns}
\usetikzlibrary{trees}
\usetikzlibrary{shapes}
\usetikzlibrary{calc}

\usepackage{graphicx}
\graphicspath{{figures/}}
\usepackage{booktabs}
\usepackage[font=small,labelfont=bf]{caption}
\usepackage{amsfonts, amsmath, amsthm, amssymb}
\usepackage{wrapfig}

\begin{document}

\begin{minipage}[b]{0.7\linewidth}
\veryHuge \color{NavyBlue} \textbf{Building Game Theoretical Software in a Research Environment} \color{Black}\\ % Title
\Huge\textit{And applying it to healthcare modelling}\\[1cm] % Subtitle
\huge \textbf{James Campbell \& Dr Vince(nt) Knight}\\ % Author(s)
\huge Department of Mathematics\\ % University/organization
\end{minipage}
%
%
\begin{minipage}[b]{0.3\linewidth}
\centering
\includegraphics[width=15cm]{images/logo.eps}
\end{minipage}

\vspace{1cm}
\begin{multicols}{2}


\begin{figure}[H]
    \begin{center}
        \tikzstyle{level 1}=[level distance=7cm, sibling distance=5cm]
        \tikzstyle{level 2}=[level distance=7cm, sibling distance=5cm]
        \tikzstyle{player} = [text width=5em, draw, text centered, rectangle, fill=blue!20, inner sep=1pt]
        \tikzstyle{nature} = [minimum width=3pt,circle,  draw, fill=red!20, inner sep=1pt]
        \tikzstyle{end} = [circle, minimum width=3pt, fill, inner sep=0pt, right]
        \tikzstyle{corner} = [inner sep=0pt, outer sep=0pt, draw,circle]
        \begin{tikzpicture}[grow=right, sloped, scale=.6]
            \node[player]{Hospital 1}
            child {node [corner] (A1) {} node [right=2pt]{$C^{(1)} - 1$} }
            child {node [corner] (A2) {} node [right=2pt] {$1$} edge from parent node[above] {$\color{blue}{c^{(1)}}$}} ;
            \draw [bend right, dashed] (A1) to (A2);
            \node (B3) [player,right=3pt] at ($(A1)!0.5!(A2)$) {Hospital 2}
            child {node [corner] (A3) {} node [right=2pt]{$C^{(2)} - 1$} }
            child {node [corner] (A4) {} node [right=2pt] {$1$} edge from parent node[above] {$\color{red}{c^{(2)}}$}} ;
            \draw [bend right] (A3) to (A4);
            \node (e) [player,right=3pt] at ($(A3)!0.5!(A4)$) {Ambulance}
            child {node [corner] (A5) {} node [right=2pt]{$\Lambda$} }
            child {node [corner] (A6) {}  node [right=2pt] {$0$}edge from parent node[above] {$\lambda_1$}} ;
            \draw [bend right] (A5) to (A6);
            \node (e) [end,right=7pt] at ($(A5)!0.5!(A6)$) {};
            \node [right] at (e) {$(|u_1^{(1)}-u_2^{(1)}|,|u_1^{(2)}-u_2^{(2)}|,|w^{(1)}-w^{(2)}|)$};
        \end{tikzpicture}
        \caption{Underlying Stackelberg Game}\label{fig:stackelberg_game}
    \end{center}
\end{figure}


\color{Goldenrod}
\section*{Stackelberg game, MC, NFG}
The issue of waiting times for ambulances at at two hospitals can be modelled as a simple Stackelberg game where each hospital has its own AE and Ward.
Patients arrive at the AE at rate $\lambda$ and if there is space in the queue they join it.
If there is no space in the queue that patient is lost.
Each patient has an AE service time, $\mu$, which represents how long their treatment in AE will last.
A proportion, $p$, of patients are then dismissed immediately.
Those who are not dismissed are admitted to the ward if there is space, otherwise they will wait in AE, continuing to occupy a bed.
Once admitted, they are treated in the ward with a service time $\hat{\mu}$ and then dismissed without delay.


\color{Brown}
\section*{Sage, Open Source Software}
Sage: "Creating a viable free open source alternative to Magma, Maple, Mathematica and Matlab".
Ref?
The areas of Game Theory that we decided to implement in Sage were Matching Games, Co-operative Games and Normal Form Games.
Matching Games allow us to solve problems where players need to be paired with each other, but they have their own prefences.
We normally look for stable matching where no player has any incentive to change their pairing.
Co-operative Games are used in situations where players each contribute to a system and those seperate contributions require their own payoff.
Normal Form Games involve players choosing different strategies against each other and obtaining a payoff.
Nash equilibria occur when no player has any incentive to change which strategy they play.


\color{Olive}
\section*{Limitations of MC}
Fusce magna risus, molestie ut porttitor in, consectetur sed mi. Vestibulum ante ipsum primis in faucibus orci luctus et ultrices posuere cubilia Curae; Pellentesque consectetur blandit pellentesque. Sed odio justo, viverra nec porttitor vel, lacinia a nunc. Suspendisse pulvinar euismod arcu, sit amet accumsan enim fermentum quis. In id mauris ut dui feugiat egestas. Vestibulum ac turpis lacinia nisl commodo sagittis eget sit amet sapien. Phasellus imperdiet, tortor vitae congue bibendum, felis enim sagittis lorem, et volutpat ante orci sagittis mi. Morbi rutrum laoreet semper. Morbi accumsan enim nec tortor consectetur non commodo nisi sollicitudin. Proin sollicitudin. Pellentesque eget orci eros. Fusce ultricies, tellus et pellentesque fringilla, ante massa luctus libero, quis tristique purus urna nec nibh. Proin sollicitudin. Pellentesque eget orci eros. Fusce ultricies, tellus et pellentesque fringilla, ante massa luctus libero, quis tristique purus urna nec nibh.


\color{SteelBlue}
\section*{Markov Decision Process}
A Markov Decision Process is a model of decision making in a dynamic framework in which a decision maker makes decisions based on a particular system state \cite{puterman2009markov}.
The process is that an agent starts off in a particular state, makes a decision, and then has a probabilistic transition whose probabilities may depend on the previous decision.

\section*{Q-Learning}
Q-learning is the process of assigning a state-action value or Q-value to the combination of being in a state, taking an action and observing a reward.
The Q-value is then updated by assessing the maximum value of being in the new state.
The higher the Q-value the more likely a player is to choose action $a$ when in state $s$.

\end{multicols}
\end{document}